\thispagestyle{empty}
\vspace*{1.0cm}

\begin{center}
    \textbf{Abstract}
\end{center}

\vspace*{0.5cm}

\noindent
This template is intended to give an introduction of how to write diploma and master thesis at the chair 'Architektur der Vermittlungsknoten' of the Technische Universität Berlin. Please don't use the term 'Technical University' in your thesis because this is a proper name. 
\\
\\
On the one hand this PDF should give a guidance to people who will soon start to write their thesis. The overall structure is explained by examples. On the other hand this text is provided as a collection of LaTeX files that can be used as a template for a new thesis. Feel free to edit the design.
\\
\\
It is highly recommended to write your thesis with LaTeX. I prefer to use Miktex in combination with TeXnicCenter (both freeware) but you can use any other LaTeX software as well. For managing the references I use the open-source tool jabref. For diagrams and graphs I tend to use MS Visio with PDF plugin. Images look much better when saved as vector images. For logos and 'external' images use JPG or PNG. In your thesis you should try to explain as much as possible with the help of images.
\\
\\
The abstract is the most important part of your thesis. Take your time to write it as good as possible. Abstract should have no more than one page. It is normal to rewrite the abstract again and again, so  probaly you won't write the final abstract before the last week of due-date. Before submitting your thesis you should give at least the abstract, the introduction and the conclusion to a native english speaker. It is likely that almost no one will read your thesis as a whole but most people will read the abstract, the introduction and the conclusion.
\\
\\
Start with some introductionary lines, followed by some words why your topic is relevant and why your solution is needed concluding with 'what I have done'. Don't use too many buzzwords. The abstract may also be read by people who are not familiar with your topic.